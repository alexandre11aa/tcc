\chapter*{LISTA DE ABREVIATURAS E SIGLAS}
%\addcontentsline{toc}{chapter}{LISTA DE ABREVIATURAS E SIGLAS}
\thispagestyle{empty}
\begin{table}[h!]
	\flushleft
	\renewcommand{\arraystretch}{1.8}
	\begin{tabular}{lcl}
AESA && Agência Executiva de Gestão das Águas do Estado da Paraíba\\
AD &&Anderson-Darling\\
CETESB &&Companhia Ambiental do Estado de São Paulo\\
COBYLA &&\textit{Constrained Optimization By Linear Approximations}\\
CG &&\textit{Conjugate Gradient}\\
CSV &&\textit{Comma-Separated-Values}\\
DA &&\textit{Dual-Annealing}\\
DAEE &&Departamento de Águas e Energia Elétrica\\
DE &&\textit{Differential Evolution}\\
IDF &&Intensidade-Duração-Frequência\\
IDW &&Ponderação do inverso da distância\\
INPE &&Instituto Nacional de Pesquisas Espaciais\\
KS &&Kolmogorov-Smirnov\\
L-BFGS-B &&Broyden-Fletcher-Goldfarb-Shanno\\
LM &&Levenberg-Marquardt\\
MMQ &&Método dos mínimos quadrados\\
MK &&Mann-Kendall\\
MVS &&Máxima verossimilhança\\
NM &&Nelder-Mead\\
NS &&Nash–Sutcliffe\\
NWS &&\textit{National Weather Service}\\
RMSE &&Raiz do Erro Quadrático Médio\\
PDF &&\textit{Portable Document Format}\\
TNC &&\textit{Truncated} Newton\\

\end{tabular} 
\end{table}
\newpage

\chapter*{LISTA DE SÍMBOLOS}
%\addcontentsline{toc}{chapter}{LISTA DE SÍMBOLOS}
\thispagestyle{empty}

\noindent \textbf{Símbolos do Alfabeto Grego}

\begin{table}[h!]
	\flushleft
	\renewcommand{\arraystretch}{1.8}
	\begin{tabular}{lp{.3cm}l}
$ \alpha \quad\quad\quad\quad \  $ && Parâmetro de escala de Kolmogorov-Smirnov \\
$ \beta_{1} $ && Parâmetro de escala da distribuição de Gumbel \\
$ \beta_{2} $ && Parâmetro de posição de Kolmogorov-Smirnov \\
$ \gamma $ && Constante de Euler-Mascheroni \\
$ \delta $ && Desvio padrão \\
$ \mu $ && Moda de Gumbel \\
$ \pi $ && Constante pi \\
\end{tabular} 
\end{table}

\noindent \textbf{Símbolos do Alfabeto Latino}

\begin{table}[h!]
	\flushleft
	\renewcommand{\arraystretch}{1.8}
	\begin{tabular}{lp{.3cm}l}
$ a $, $ b $, $ c $, $ d $ && Parâmetros da equação da intensidade das chuvas \\
$ a_1 $, $ b_1 $, $ a_2 $, $ b_2 $ && Ajustes matemáticos do método dos mínimos quadrados \\
$ C $ && Coeficiente de desagregação \\
$ c_i $ && Parâmetro $c$ de um determinado período de retorno \\
$ Crit $ && Valor crítico da estatística \\
$ Crit_1 $ && Valor crítico da estatística de 1\% \\
$ Crit_5 $ && Valor crítico da estatística de 5\% \\
$ D $ && Distância entre os dois pontos \\
$ Dn $ && Diferença \\
$ Dn_{max} $ && Diferença máxima \\
$ e $ && Constante de Euler \\
$ F(Xt) $ && Função cumulativa de Gumbel \\
$ Fr_{Exced} $ && Frequência excedida da distribuição analisada \\
$ Fr_{n\_Exced} $ && Frequência não excedida da distribuição analisada \\
$ Fr $ && Frequência da distribuição analisada \\
\end{tabular} 
\end{table}

\newpage

\thispagestyle{empty}

\begin{table}[h!]
	\flushleft
	\renewcommand{\arraystretch}{1.8}
	\begin{tabular}{lp{.3cm}l}
$ I \quad\quad\quad\quad \ \ \ $ && Intensidade \\
$ i $ && Posição do ano precipitado \\
$ i_1 $ && Intensidade com menor duração do tempo de retorno utilizado \\
$ i_2 $ && Intensidade com maior duração do tempo de retorno utilizado \\
$ i_3 $ && Intensidade resultante \\
$ n $ && Número total de algum item \\
$ P $ && Potência elevada da distância \\
$ P(x_i ; y_i) $ && Precipitação a ser descoberta \\
$ P(y_i ; x_i) $ && Precipitação já existente \\
$ Pr $ && Precipitação desagregada \\
$ t $ && Duração \\
$ Tr $ && Tempo de retorno \\
$ t_1 $ && Tempo de menor duração \\
$ t_2 $ && Tempo de maior duração \\
$ t_3 $ && Tempo resultante \\
$ w $ && Peso do ponto amostral \\
$ W_j $ && Peso do ponto amostral das precipitações existentes \\
$ \bar{X} $ && Média das precipitações diárias anuais máximas \\
$ x $, $ y $ && Incógnitas que representam pontos das funções linear e de potência \\
$ x_i $ && Latitude do ponto que se quer descobrir a precipitação \\
$ \bar{X_i} $ && Precipitação diária anual máxima \\ 
$ x_j $ && Latitude do ponto próximo \\
$ Xt $ && Mediana de Gumbel \\
$ y_i $ && Longitude do ponto que se quer descobrir a precipitação \\
$ y_j $ && Longitude do ponto próximo \\
\end{tabular} 
\end{table}