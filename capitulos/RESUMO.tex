\thispagestyle{empty}

\singlespacing
\begin{center}
	\textbf{RESUMO}
\end{center}

\noindent O resumo é uma apresentação concisa dos pontos relevantes do documento. Embora não seja um elemento obrigatório em projetos de pesquisa, optou-se por mantê-lo neste modelo por ser necessário este item no acompanhamento dos Trabalhos de Conclusão de Curso do UNIFIP e como ferramenta para que o aluno pratique para a elaboração deste na versão final da pesquisa. O resumo é constituído de uma sequência de frases concisas e objetivas, fornecendo uma visão rápida e clara do conteúdo do projeto. O texto deve ser redigido em parágrafo único, com no máximo 500 palavras em espaçamento simples e seguido das palavras representativas do conteúdo do estudo, isto é, palavras-chave, em número de três a cinco, separadas entre si por ponto e vírgula e finalizadas por ponto. No resumo deve-se inserir brevemente uma introdução/contextualização sobre o assunto, o objetivo do projeto, metodologia a ser aplicada e resultados esperados e/ou parciais.

\ \

\noindent \textbf{Keywords:} palavra 1; palavra 2; palavra 3; palavra 4; palavra 5. (Devem ser grafadas com as iniciais em letra minúscula, com exceção dos substantivos próprios e nomes científicos. As palavras devem ser separadas por ponto e vírgula e finalizadas por ponto). Evite repetir palavras já constantes do título da pesquisa. A ideia é que as palavras-chave tragam informações adicionais da sua pesquisa.
\newpage