\thispagestyle{empty}

\singlespacing
\begin{center}
	\textbf{RESUMO}
\end{center}

\noindent Compreender o comportamento das precipitações pluviais intensas de uma região através da equação das chuvas é uma etapa importante do desenvolvimento de projetos e estudos que envolvem a análise, e controle das águas pluviométricas. As operações que conduzem a equação citada envolvem o entendimento da relação entre intensidade, duração e frequência, além de cálculos que passam por tratamento de dados de séries históricas, métodos de otimização, distribuições probabilísticas, e testes estatísticos de aderência, tendência e acurácia. Devido a complexidade que há na compreensão e determinação dessas equações, correspondente aos cálculos e análises probabilísticas extensas, o presente trabalho se propôs a desenvolver uma ferramenta computacional com interface gráfica de código aberto. Ela permite aos projetistas e estudiosos que necessitam da intensidade das chuvas, calcularem a equação desta para regiões escolhidas a partir de suas próprias bases de dados, ou estudarem os algoritmos e o embasamento teórico que conduzem todo o processo de cálculo.

\ \

\noindent \textbf{Palavras chave:} Ferramenta computacional; Desenvolvimento em Python; Equação das chuvas intensas; Intensidade-Duração-Frequência.
\newpage