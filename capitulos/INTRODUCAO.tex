\onehalfspacing
\chapter{INTRODUÇÃO}

Ao reparar a chuva, não se considera tamanha complexidade concebida por um ser superior, devido da naturalidade em seu uso cotidiano. Porém a ausência de entendimento sobre esses eventos climáticos, em especial as precipitações pluviométricas de nível extremo, provoca catástrofes de ordem natural e antrópica, que impactam negativamente a sociedade. Isto posto, cabe aos profissionais que se dedicam aos estudos hidrológicos, desenvolverem os conhecimentos necessários para a resolução dos problemas causados pelas chuvas intensas.

Wilken (1978) deixa claro que, quanto ao projetista de águas pluviais, a parte que lhe interessa do ciclo hidrológico é justamente a precipitação de chuvas. Nesta área, métodos foram desenvolvidos para às previsões dos eventos citados, e aliado à tecnologia, a precisão atinge níveis cada vez maiores. Sobre isto, Bertoni e Tucci (2012) declaram que projetos pautados no movimento da água da chuva, como o dimensionamento de bueiros, vertedores de barragem, galerias pluviais, entre outros, necessitam de grandezas que caracterizam a precipitação pluvial máxima. Elas são conhecidas como a intensidade, duração e frequência.

A necessidade de expressar essas grandezas de forma matemática, para assim tornar possível a análise desse tipo fenômeno pluviométrico, proporcionou o desenvolvimento da equação da intensidade-duração-frequência (IDF), que é uma função que estima a intensidade da chuva. Ela é descrita por Chow, Maidment e Mays (1988) como a taxa média de precipitação em unidade de comprimento por unidade de tempo para uma determinada bacia ou sub-bacia de drenagem. Sendo necessária para o dimensionamento dos projetos citados, ela ajuda a compreender e sanar problemas causados pelas chuvas intensas. Entende-se então que a possibilidade de elaborar e fazer estudos sobre as equações IDF, de maneira mais acessível e confiável, se torna extremamente necessária para a sociedade. 

Todavia, é notada a carência de dispositivos elaborados e disponíveis para esses fins. Entendendo esses problemas, o presente trabalho se justifica por desenvolver uma ferramenta computacional simples e intuitiva, para auxiliar na determinação das equações das chuvas intensas. Ela servirá de auxílio aos projetistas que contarão com mais um instrumento para aumentar a precisão dos seus dimensionamentos. Também terá o intuito de facilitar estudos sobre o tema, e por isso, após desenvolvida será disponibilizada em código aberto para análises e possíveis aprimoramentos.
