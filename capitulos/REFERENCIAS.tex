%\chapter{REFERÊNCIAS}
\chapter*{REFERÊNCIAS}
\addcontentsline{toc}{chapter}{REFERÊNCIAS}

\begin{flushleft}
\singlespacing
ALLISON, P. D. Missing Data. Sage University Papers Series on Quantitative Applications in the Social Sciences. 07-136. Thousand Oaks, CA: Sage. 2001. Páginas 1 e 2.

\

AMERICAN NATIONAL STANDARDS INSTITUTE. C Programming Language. Disponível em: <https://www.ansi.org/> Acesso em: 05 de setembro de 2023.

\

ASSOCIATION FOR COMPUTING  MACHINERY. Fortran. Disponível em: <https://www.acm.org/> Acesso em: 05 de setembro de 2023.

\

BACK, A. J.; WILDNER, L. P. Equação de chuvas intensas por desagregação de precipitação máxima diária para o estado de Santa Catarina. Agropecuária Catarinense, Florianópolis, v.34, n.3, p. 43-47, 2021. Página 43.

\

BERTONI, J. C.; TUCCI, C. E. M. Hidrologia ciência e aplicação. Universidade Federal do Rio Grande do Sul, 4, 2012. Páginas 180-182, 200-203 e 207.

\

BESKOW, L; CORRÊA, L. L.; MAHL, M.; SIMÕES, M. C.; CALDEIRA, T. L.; NUNES, G. S.; LUCAS, E. H.; FARIA, L. C.; MELLO, C. R. Desenvolvimento de um sistema computacional de aquisição e análise de dados hidrológicos. In: XX Simpósio Brasileiro de Recursos Hídricos, 2013, Bento Gonçalves, RS. XX Simpósio Brasileiro de Recursos Hídricos – Água-Desenvolvimento Econômico e Socioambiental, 2013. Página 4.

\

BOYER, C. B. História da Matemática. 2. ed. São Paulo Edusp; Edgar Blucher Ltda, 1974. Página. 367.

\

CECILIO, R. A.; PRUSKI, F.F. Interpolação dos parâmetros de equações de chuvas intensas com uso do inverso de potências da distância. Revista Brasileira de Engenharia Agrícola e Ambiental, v.7, n.3, p.501-504, 2003. Página 501.

\

CHOW, V.; MAIDMENT, D.; MAYS, L. Applied Hydrology. McGraw-Hill Book Company, New York, 1988. Página 499.

\

COLLISCHONN, W.; DORNELLES, F. Hidrologia para engenharia e ciências ambientais. Edição 2 - revisada e ampliada. ed. Porto Alegre: Associação Brasileira de Recursos Hídricos, 2015. Páginas 13, 54, 56.

\

DAEE; CETESB. Drenagem Urbana 2a ed. São Paulo. 1980. Página 22.

\

FUNDAMENTAL ALGORITHMS FOR SCIENTIFIC COMPUTING IN PYTHON. SciPy. Disponível em: <https://www.scipy.org/> Acesso em: 05 de setembro de 2023.

\newpage

\

GAUSS, J. C. F. CARL FRIEDRICH GAUSS WERKE. Edição crítica (volume 12). Gottingen Academy of Sciences and HumanitiesSpringer-Verlag, 2018 [1900]. Página 201.

\

HELSEL, D. R.; HIRSCH, R. M.; RYBERG, K. R.; ARCHFIELD, S. A.; GILROY, E. J. Statistical Methods in Water Resources. Capítulo 3. In: HYDROLOGIC ANALYSIS AND INTERPRETATION. Book 4. Reston, Virginia: U.S. Geological Survey, 2020. Página 327.

\

HOSKING, J. R. M., WALLIS, J. R. Regional Frequency Analysis: An Approach Based on L-Moments, 224p. Cambridge University Press, Cambridge, Reino Unido, 1997. Página 202.

\

INNO SETUP IS A FREE INSTALLER FOR WINDOWS PROGRAMS. Inno Setup, 1997. Disponível em: <https://www.jrsoftware.org/> Acesso em: 05/09/2023.

\

INSTITUTO NACIONAL DE PESQUISAS ESPACIAIS (INPE). Glossário de Meteorologia - Diferença entre Tempo Meteorológico e Clima. Disponível em: <https://www.cptec.inpe.br/glossario.shtml> Acesso em: 18/05/2023.

\

INTERNATIONAL ORGANIZATION FOR STANDARDIZANTION. C++ Programming Language Standard. Disponível em: <https://www.iso.org/> Acesso em: 05 de setembro de 2023.

\

KIM, J.; RYU, J. H. A Heuristic Gap Filling Method for Daily Precipitation Series. Water Resources Management, 30(7), 2275–2294. 2016. Página 2286.

\

KIM, J.; RYU, J. H. Quantifying a Threshold of Missing Values for Gap Filling Processes in Daily Precipitation Series. Water Resour. Manag. 29, 4173–4184. 2015. Página 4176.

\

LANNA, A. E. Hidrologia ciência e aplicação. Universidade Federal do Rio Grande do Sul, 4, 2012. Páginas 106-154.

\

MARCONI, M. de A.; LAKATOS, E. M. Técnicas de Pesquisa. 8. ed. São Paulo: Atlas. 2017. Página 20.

\

MARTINEZ JUNIOR, F.; MAGNI, N. L. G. Equações de chuvas intensas do Estado de São Paulo. São Paulo Departamento de Águas e Energia Elétrica, Escola Politécnica da Universidade de São Paulo, São Paulo. 1999. Páginas 2 e 3.

\

NAGHETTINI, M.; PINTO, E. J. A. (2007). Hidrologia Estatística. CPRM – Serviço Geológico do Brasil, páginas 130-173, 206, 271 e 275-278.

\

NASH, J. E.; SUTCLIFFE, J. V. River Flow Forecasting through Conceptual Model. Part 1—A Discussion of Principles. Journal of Hydrology, 10, 282-290. 1970.

\

NETO, E. de P. S. Atlas Pluviométrico do Brasil; Determinação de Parâmetros Ótimos da Equação Intensidade, Duração e Frequência de Chuvas Utilizando Otimização Heurística. Catalão-GO, UFG, 2020. Página 49.

\newpage

\

NÓBREGA, A. E. L. Instalador da ferramenta computacional Intensio. Intensio, 2023. Disponível em: <https://drive.google.com/file/d/1IkApxCR6-j-2dgzLQgXkitvOrcYOiiNe/view> Acesso em: 21 de setembro de 2023.

\

NÓBREGA, A. E. L. Manual de uso da ferramenta computacional Intensio. Intensio, 2023. Disponível em: <https://drive.google.com/file/d/13d4H1FsTQ\_nRmmNXv0ytU7gUgkqNRu21/view> Acesso em: 21 de setembro de 2023.

\

NÓBREGA, A. E. L. Repositório de código fonte da ferramenta computacional Intensio. Intensio, 2023. Disponível em: <https://www.github.com/alexandre11aa/intensio> Acesso em: 21 de setembro de 2023.

\

THE FUNDAMENTAL PACKAGE FOR SCIENTIFIC COMPUTING WITH PYTHON. NumPy. Disponível em: <https://www.numpy.org/> Acesso em: 18/05/2023.

\

PFAFSTETTER, O., Chuvas Intensas no Brasil, 2a. edição, Rio de Janeiro, DNOS, 1982, página 426.

\

PINTO, E. J. de A. Atlas Pluviométrico do Brasil; Metodologia para definição das equações Intensidade-Duração-Frequência do Projeto Atlas Pluviométrico. Belo Horizonte CPRM, 2013. Página 25.

\

PYINSTALLER BUNDLES A PYTHON APPLICATION AND ALL ITS DEPENDENCIES INTO A SINGLE PACKAGE. PyInstaller. Disponível em: <https://www.pyinstaller.org/> Acesso em: 05 de setembro de 2023.

\

PYTHON PACKAGE INDEX. PyPi, 1991. Disponível em: <https://pypi.org/> Acesso em: 05 de setembro de 2023.

\

PYTHON SOFTWARE FOUNDATION. Python.org, 1989. Disponível em: <https://www.python.org/> Acesso em: 05 de setembro de 2023.

\

SANTOS, V. C. Probabilidade de ocorrência de chuvas e sua variação espacial e temporal na Bacia Hidrográfica do Rio Tapajós. Dissertação de Mestrado – Universidade Federal do Pará. Instituto de Tecnologia. Programa de Pós-Graduação em Engenharia Civil, Belém, 2017.

\

SILVA, A. S. A.; STOSIC, B.; MENEZES, R. S. C.; SINGH, V. P. Comparison of interpolation methods for spatial distribution of monthly precipitation in the state of pernambuco, Brazil. 2019. Páginas 7 e 9.

\

SILVEIRA, A. L. L. da. Hidrologia ciência e aplicação. Universidade Federal do Rio Grande do Sul, 4, 2012. Página 36.

\

SILVEIRA, André. RBRH - Revista Brasileira de Recursos Hídricos, Volume 5 n.4, páginas 143-147. 2000. Página 45.

\newpage

\

TKINTER - PYTHON INTERFACE TO TCL/TK. Python.org, 1989. Disponível em: <https://docs.python.org/3/library/tkinter.html> Acesso em: 05 de setembro de 2023.

\

WILKEN, P. S. Engenharia de drenagem superficial. São Paulo, Companhia de Tecnologia de Saneamento Ambiental, 1978. Páginas 1-2, 23, 50-51.

\

WISLER, C.O.; BRATER, E.F. Hydrology. 2nd Edition, John Wiley and Sons Inc., New York, 1959.

\

XAVIER, A. C.; KING, C. W.; SCANLON, B. R. Daily gridded meteorological variables in Brazil (1980–2013). International Journal of Climatology, Wiley Online Library, v. 36, n. 6, p. 2644–2659, 2016. Página 2658.
\end{flushleft}