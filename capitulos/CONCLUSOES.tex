\chapter{CONCLUSÕES}

Entendendo tudo que envolve o desenvolvimento de uma equação IDF, fica claro que para chegar a um resultado consistente é necessário a compreensão de diversos fatores que vão além de seu cálculo. Entretanto, diante de tudo que foi criado, entende-se que a ferramenta cumpre seu dever no que se propõe, que é fornecer uma base sólida e confiável para o projetista que pretendem desenvolver a equação para um local específico. 

Destaca-se que o programa não irá isentar o calculista da prévia análise da consistência dos dados de séries históricas que venha a ser usada como fonte, como também as particularidades do local alvo, tendo ele responsabilidade de uso. Contudo, a ferramenta lhe oferecerá um ambiente propício para que seja viável o máximo de cálculos e análises em torno da equação IDF do lugar pretendido.

Indo além do caráter prático da ferramenta, é importante salientar toda a pesquisa, tratamento de dados e algoritmos que foram desenvolvidos em torno da equação para que o programa se tornasse viável. Isso porque todos eles são públicos, e por isso, passíveis de análise, estudo, melhora, modificação e evolução por parte daqueles que se interessam pelo tema.

Assim sendo, foi concluída a criação de uma ferramenta computacional que tem como pretensão servir de ajuda aos projetistas e estudiosos, tendo como características a sua simplicidade de uso, e facilidade de acesso em todos os seus âmbitos.
